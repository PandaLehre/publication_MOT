\documentclass[lettersize,journal]{IEEEtran}
\usepackage{amsmath,amsfonts}
\usepackage{algorithmic}
\usepackage{algorithm}
\usepackage{array}
\usepackage[caption=false,font=normalsize,labelfont=sf,textfont=sf]{subfig}
\usepackage{textcomp}
\usepackage{stfloats}
\usepackage{url}
\usepackage{verbatim}
\usepackage{graphicx}
\usepackage{cite}
\hyphenation{op-tical net-works semi-conduc-tor IEEE-Xplore} 
% updated with editorial comments 8/9/2021


\begin{document}

\title{Machine learning based multi-object tracking without dynamic models and hard association metrics}

\author{Christian Alexander Holz, Christian Bader, Matthias Drüppel
	\thanks{C. Holz is with Daimler Truck AG, Research and Advanced Development, Stuttgart, Germany}
	\thanks{C. Bader is with Daimler Truck AG, Research and Advanced Development, Stuttgart, Germany}
	\thanks{M. Drüppel is with the Center for Artificial Intelligence,
		Duale Hochschule Baden-Württemberg (DHBW), Stuttgart, Germany}
}

% The paper headers
\markboth{Journal of \LaTeX\ Class Files,~Vol.~tbd, No.~tbd, tbd~2024}%
{Shell \MakeLowercase{\textit{et al.}}: A Sample Article Using IEEEtran.cls for IEEE Journals}

% \IEEEpubid{0000--0000/00\$00.00~\copyright~2021 IEEE}% Remember, if you use this you must call \IEEEpubidadjcol in the second% column for its text to clear the IEEEpubid mark.
\maketitle


\begin{abstract}
    In this paper, we develop Machine Learning (ML)-based methods for Multi Object Tracking (MOT)
    within the context of Advanced Driver Assistance Systems (ADAS).
    Given the increasing complexity and demand for precise and efficient object tracking systems
    in the automotive industry, this work focuses on the integration of ML techniques into
    established tracking methodologies.
    Key contributions encompass the creation and evaluation of three specialized neural networks: (i)
    the Single Prediction Network (SPENT) for predicting the trajectories of tracked objects,
    (ii) the Single Association Network (SANT) for associating incoming sensor objects
    with existing tracks, (iii) and the Multi Association Network (MANTa) for associating
    multiple sensor objects with existing tracks.
    Figure \ref{fig1} provides an overview of our approach (i) and (ii).
    These networks aim to combine ML methods with a traditional Kalman filter framework,
    offering a data driven approach to addressing MOT challenges.
    We integrate our three ML networks into a Kalman framework and evaluate the performance,
    both of the components itself and the overall system.
    By replacing single components, we get a clearer understanding of the impact of the ML models
    on the overall tracking system.
    This approach also leaves the modularity of system intact while enabling machine learning (ML)
    for certain tracking tasks.
    The results reveal a modular, robust, and maintainable tracker,
    underscoring the potential of ML integration in ADAS.
    % hier müssen später noch unbedingt die wichtigsten Ergebnisse rein.
\end{abstract}


\begin{IEEEkeywords}
	Article submission, IEEE, IEEEtran, journal, \LaTeX, paper, template, typesetting.
\end{IEEEkeywords}


\section{Introduction}
% MD: Ich würde erst ganz allgemein bleiben und auch nochmal sagen, welche Schwierigkeiten klassisches Tracking hat, z.B., dass spezifische Lösungen für spezifische Szenarien entwickelt werden müssen. Diese Entwicklung ist aber nicht direkt Daten getrieben und führt zu einer hohen Anzahl an Parametern, die eingestellt werden müssen und einer schlechten Wartbarkeit und Anpassbarkeit auf andere Fahrzeug Modelle oder Verkehrsituationen. Data driven Ansätze können hier helfen, diese Probleme zu lösen.
% MD: Danach würde ich über die einzelnen Punkte sprechen: Erst prediction, dann Assoziation. Die kriegen auch jeweils ihren eigenen Absatz. Und dann noch einen Absatz für alles zusammen. Oder erst alles zusammen und dann einzeln, aber ganz klar getrennt.
%%%%%%%%%%%%%%%%%%%%%% gedanklicher Absatztitel: Allgemeine Einführung MOT
\IEEEPARstart{T}{he}
ongoing evolution of Advanced Driver Assistance Systems (ADAS) has brought the need for precise
and reliable Multi Object Tracking (MOT) into the spotlight 
\cite{KF_simple_cues.2022} \cite{KF_simple_online_realtime.2016}
\cite{DL_CNN_mot_sot_based.2017} 
\cite{DL_RNN_mot.2016} \cite{DL_RNN_data_association.2019}
\cite{DL_ATT_CNN_soda.2020} \cite{DL_ATT_CNN_mot_sot_based.2017}
.
In complex and dynamic environments, as encountered in urban traffic, it is crucial to
simultaneously and accurately capture the positions and movements of multiple objects.
Tracking multiple objects is a key challenge in computer vision (CV).

In the commonly used Tracking-by-Detection (TbD) paradigm,
a tracker fuses detected sensor objects (SO) to create consistent object tracks over time.
A key challenge within this paradigm is associating incoming measurements SO
with their corresponding existing object tracks or initializing new object tracks.

Offline methods \cite{offline_mot.2017} process the entire video material at once
in a batch process. However, these methods are unsuitable for most real-time applications,
such as ADAS.
In such applications, it is crucial to predict the state of objects immediately 
after new detections.
Therefore, most recent approaches for tracking multiple objects rely on online methods
that do not depend on future image information.
Online methods use various features to estimate the similarity between the recognised objects
and the existing tracks.
This can be done on the basis of predicted positions or similarities in appearance
\cite{KF_simple_online_realtime.2016}
\cite{DL_CNN_mot_sot_based.2017} 
\cite{DL_RNN_mot.2016} \cite{DL_RNN_data_association.2019}
.
While some approaches only consider the most recent recognition corresponding to a track,
other methods integrate temporal information into a track history.
For example, methods use recurrent neural networks 
\cite{DL_RNN_mot.2016} \cite{DL_RNN_data_association.2019}
or attention mechanisms
\cite{DL_ATT_CNN_soda.2020} \cite{DL_ATT_CNN_mot_sot_based.2017}
to aggregate temporal information.

%%%%%%%%%%%%%%%%%%%%%% gedanklicher Absatztitel: prediction
For state prediction, in many TbD approaches, Kalman filters and their variants 
have proven to be effective.
However, they reach their limits in more complex scenarios, particularly in the presence
of non-linear motion patterns and interactions among multiple objects.
In this work, we introduce a novel MOT approach that leverages Machine Learning (ML)
to overcome these challenges.
We specifically focus on the development and implementation of Neural Networks (NN),
which can enable more precise and flexible data-driven object tracking without
relying on cumbersome heuristics and hyperparameters.

%%%%%%%%%%%%%%%%%%%%%% gedanklicher Absatztitel: association
The data association is typically carried out based on similarity scores calculated between
the measurements and the existing tracks.
These similarity scores may rely on the latest detection or be aggregated from 
historical detections.
As in Mertz et al \cite{DL_RNN_data_association.2019}, the aim of this work was to develop
a data-based approach that can learn to completely solve the combinatorial 
Non Deterministic Polynomial Time (NP) hard optimisation problem of data association.
Mertz et al \cite{DL_RNN_data_association.2019} use a distance matrix based on the 
Euclidean distance measure as input data for the developed DA network,
thus replacing an association algorithm such as the Hungarian Algorithm (HA).
It can be assumed that the Euclidean distance measure was used as the basis for calculating
the ground truth (GT) training data (distance matrices) and for the evaluation.
However, this is not explicitly stated.
It can therefore be argued that this calculation step deprives the network 
of the opportunity to follow a different association logic or to learn it on the basis of data.
In the context of this work, the hypothesis was therefore put forward that a Gated 
Recurrent Unit (GRU)-based association network can be formed by an undefined distance measure 
to increase the assignment of the temporal memory component and thus of the history. 

The aim of this work was therefore to develop an association network that is intended to solve
the assignment of one or more sensor objects (SO) to an existing number of object tracks
without a defined distance measure.


\section{Related work}
%%%%%%%%%%%%%%%%%%%%%% gedanklicher Absatztitel: Einführung unser Ansatz 
Our primary contribution is the development and evaluation of three NN that we labeled:
(i) the Single Prediction Network (SPENT), (ii) the Single Association Network (SANT),
and (iii) the Multi Association Network (MANTa).
Figure \ref{fig1} provides an overview of our approach (i) and (ii).
The proposed Sinlge Prediction Network (SPENT) processes the sensor objects (SO) per time step,
which consists of a state vector and contains information such as object position,
orientation and dimension.
SPENT predicts a fixed-dimensional state vector for each SO based on
the received state vectors per SO.
The output predictions from SPENT are used as input to the Single Association Network (SANT)
to build target trajectories or object tracks.
In the remainder of the following sections, we will explain the proposed modules in detail.
\begin{figure}[htbp]
	\centerline{\includegraphics[width=1.0\linewidth]{figs/MOT_Framework_Integration_SPENT_SANT.png}}
	\caption{Schematic representation of the two integrated networks SPENT and SANT in a tracking-by-detection (TbD) framework.}
	\label{fig1}
\end{figure}

In comparison to the Kalman filter, SPENT is capable of predicting the state of 
individual objects without the need for a predefined state or observation model at runtime.
SPENT holds the potential, particularly in terms of adaptability to various scenarios
and the ability to effectively handle nonlinearities.
Many conventional tracking systems rely on static methods for data association,
often based on simple heuristics or fixed thresholds.
In contrast, SANT employs machine learning to automate these processes and adapt
more effectively to different scenarios. 
As a result, within the TbD MOT framework,
SANT replaces the calculation of a distance metric and the Hungarian algorithm for
the corresponding assignment.
Furthermore, we integrate SPENT and SANT into an existing tracking system and demonstrate 
their performance through multiple tests and comparisons with established methods.
% Ich wäre mit Wertungen immer sehr vorsichtig in Papern. Also hier schreibst du "valuable insights". Ich würde eher ein nicht wertendes adjectiv suchen. Ob das wertvoll ist oder nicht, müssen andere entscheiden.
% --> CH: exciting? 
This work provides exciting insights and a significant advancement in the
development of Advanced Driver Assistance Systems (ADAS),
contributing to the further evolution of technologies for autonomous driving (AD).

\section{Tracking with Prediction and Association Networks}
Wir wenden das Paradigma des Tracking-by-Detection (TbD) an, bei dem ein Tracker 
die Objekterkennungen fusioniert, um Objektspuren zu erzeugen, die über die Zeit konsistent sind.
Ba-Tuong Vo et al. [xx] stellt beispielsweise einen Framework für die Untersuchung 
von Tracking Ansätzen zur Verfügung, welche dem TbD Paradigma folgen.
Wir schlagen einen Tracker vor mit jeweils einem Long Short-Term Memory (LSTM) Netzwerk 
für die Prädiktion und Assoziation der Sensorobjekte (SO) zu den bestehenden Tracks.

%%%%%%%%%%%%%%%%%%%%%% gedanklicher :) Absatztitel: SPENT
\subsection{Single Prediction Network (SPENT)}
Die meisten bestehenden Verfolgungsmethoden verknüpfen eingehende Erkennungen paarweise mit Objektzuständen, die durch ein einfaches Bewegungsmodell, z. B. ein Modell mit konstanter Geschwindigkeit, unter Verwendung eines Kalman-Filters vorhergesagt werden. Neuere Arbeiten haben jedoch gezeigt, dass die Aggregation zeitlicher Informationen sowie von Kontextinformationen die Verfolgung mehrerer Objekte verbessern kann, indem zusätzlich zu den paarweisen Ähnlichkeiten zwischen den Erkennungen Informationen höherer Ordnung genutzt werden [xx].
Unserer Ansatz sieht es vor, auf die Bewegungsmodelle zu verzichten und stattdessen die Hiddenstates der LSTM Schicht als objektspezifisches Parameterset zu nutzen.
Die initialen Werte der Hiddenstates der LSTM Schicht werden im Tracking anhand der erhaltenen Messdaten aktualisiert.
Durch diese Aktualisierung erfolgt somit eine interne Korrektur über den Sequenzverlauf.

%%%%%%%%%%%%%%%%%%%%%% gedanklicher :) Absatztitel: SANT
\subsection{Single Association Network (SANT)}

%%%%%%%%%%%%%%%%%%%%%% gedanklicher :) Absatztitel: MANTa
\subsection{Multi Association Network (MANTa)}


\section{Experimental Evaluation}
... KITTI-Car Benchmark.


\section{Conclusion}
The conclusion goes here.


\section*{Acknowledgments}
This should be a simple paragraph before the References to thank those individuals and institutions who have supported your work on this article.


{\appendix[Proof of the Zonklar Equations]
\section*{Proof of the First Zonklar Equation}
Appendix goes here.
\section*{Proof of the Second Zonklar Equation}
And here.
}

\bibliographystyle{IEEEtran}
\bibliography{IEEEabrv, ./lit.bib}

\vfill

\end{document}


